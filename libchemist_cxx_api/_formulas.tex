\documentclass{article}
\usepackage{epsfig}
% Packages requested by user
\usepackage{amsmath}

\pagestyle{empty}
\begin{document}
$n_0$
\pagebreak

$n_0 -1$
\pagebreak

$n_1$
\pagebreak

$n_0 + n_1 - 1$
\pagebreak

$n_{0}$
\pagebreak

$n_{0} -1$
\pagebreak

$n_{1}$
\pagebreak

$n_0 + n_1 -1$
\pagebreak

$N$
\pagebreak

$\chi(\vec{r};\vec{c}, \vec{\alpha})$
\pagebreak

$i$
\pagebreak

$g(\vec{r};c, \alpha)$
\pagebreak

$c$
\pagebreak

$\alpha$
\pagebreak

$\vec{r}$
\pagebreak

$\ell$
\pagebreak

$2\ell + 1$
\pagebreak

$\binomial{2 + \ell}{2}$
\pagebreak

$x^ay^bz^c$
\pagebreak

$a+b+c = \ell$
\pagebreak

$N+1$
\pagebreak

$T$
\pagebreak

$U$
\pagebreak

$r$
\pagebreak

$R$
\pagebreak

$A$
\pagebreak

$B$
\pagebreak

$I$
\pagebreak

$\left\mid A[I] - B[I]\right\mid@ftwo differs by less than @p atol, the difference is considered indistinguishable from zero and the elements the same. Particularly for very large elements achieving a difference on the order of @p atol is not always reasonable; in these cases, one cares more about the percent error. In this case $
\pagebreak

$ must be less than a relative tolerance, @p rtol. This function combines these criteria and returns true if $
\pagebreak

$ and $
\pagebreak

$ satisfy: @f{ \left\mid A - B\right\mid \le atol + rtol\left\mid B\right\mid @f} Note that this is **NOT** symmetric in $
\pagebreak

$, but is commensurate with $
\pagebreak

\end{document}
